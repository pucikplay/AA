\documentclass{article}

\usepackage{graphicx}
\usepackage[letterpaper,top=1cm,bottom=1cm,left=2cm,right=2cm,marginparwidth=1.75cm]{geometry}
\usepackage{amsmath}
\usepackage{polski}

\title{AA Laboratorium 7}
\author{Gabriel Budziński}
\date{June 2023}

\begin{document}

\maketitle

\section{Zadanie 15}

\subsection{Analiza teoretyczna}

Dla argumentu $n$ linia 6 wywoła się $n$ razy w pierwszym wywołaniu, a oprócz tego w wywołaniach rekurencyjnych, przez co dostajemy:
\begin{equation}
    \begin{cases}
        f_0 = 0\\
        f_n = n + \sum\limits_{i=0}^{n-1}{f_i}
    \end{cases}
\end{equation}

Każdy składnik dolnego równania wymnażamy przez $z^n$ i nakładamy sumę:

\[\sum\limits_{n \geq 0}{f_n z^n} = \sum\limits_{n \geq 0}{nz^n} + \sum\limits_{n \geq 0}{\left(\sum\limits_{i=0}^{n-1}{f_i}\right)z^n}\]

Teraz wyliczamy każdy ze składników:

\[\sum\limits_{n \geq 0}{f_n z^n} = F(z)\]

Dla drugiego korzystamy z $\textbf{Right Shift}$ oraz $\textbf{Index Multiply}$:

\[\sum\limits_{n \geq 0}{nz^n} \stackrel{RS}{=} z\sum\limits_{n \geq 0}{nz^{n-1}} \stackrel{IM}{=} z\frac{d}{dz}\left(\sum\limits_{n \geq 0}{z^n}\right) = z\frac{d}{dz}\left(\frac{1}{1-z}\right) = \frac{z}{{(1-z)}^2}\]

Dla trzeciego korzystamy z $\textbf{Right Shift}$ oraz $\textbf{Partial Sum}$:

\[\sum\limits_{n \geq 0}{\left(\sum\limits_{i=0}^{n-1}{f_i}\right)z^n} = 0 + \sum\limits_{n \geq 1}{\left(\sum\limits_{i=0}^{n-1}{f_i}\right)z^n} \stackrel{RS}{=} z\sum\limits_{n \geq 1}{\left(\sum\limits_{i=0}^{n-1}{f_i}\right)z^{n-1}} = z\sum\limits_{n \geq 0}{\left(\sum\limits_{i=0}^{n}{f_i}\right)z^n} \stackrel{PS}{=} z\frac{1}{1-z}F(z)\]

Z tego otrzymujemy

\[F(z) = \frac{z}{{(1-z)}^2} + \frac{z}{1-z}F(z)\]

Czyli

\[F(z) = \frac{z}{{(1-z)}^2} \cdot \frac{1-z}{1-2z} = \frac{z}{(1-2z)(1-z)}\]

Teraz korzystając z $\textbf{Convolution}$ oraz $\textbf{Right Shift}$

\[F(z) = z\sum\limits_{n \geq 0}{z^n} \cdot \sum\limits_{n \geq 0}{{(2z)}^n} \stackrel{C}{=} z\sum\limits_{n \geq 0}\left({\sum\limits_{k=0}^{n}{2^k1_{n-k}}}\right)z^n = z\sum\limits_{n \geq 0}{\frac{1-2^{n+1}}{1-2}z^n} = \sum\limits_{n \geq 0}{\left(2^{n+1} - 1\right)z^{n+1}} = f_0z^0 + \sum\limits_{n \geq 1}{\left(2^n - 1\right)z^n}\]

Otrzymujemy $f_n = 2^n - 1$.

\subsection{Wyniki eksperymentalne}

Wyniki sprawdzono dla $n \in \{0,\dots,32\}$ i były jednakowe z teoretycznymi.

\section{Zadanie 16}

\subsection{Analiza teoretyczna}

\subsection{Wyniki eksperymentalne}

\end{document}